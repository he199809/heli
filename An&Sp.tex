\documentclass[UTF8,12,a4paper]{ctexart}
%\documentclass[13pt]{extarticle}
\usepackage{amsmath,amsthm,amsfonts,amssymb,amscd}
\usepackage{latexsym}
\usepackage{setspace}   %设置行距的宏包
\usepackage{fancyhdr}
%\usepackage{geometry}   %设置页边距的宏包
\usepackage{titlesec}   %设置页眉页脚的宏包
\usepackage{tikz-cd}    %画交换图的宏包 
\usepackage{enumerate}
\usepackage{color}
%\usepackage{extsizes}
\usepackage{tikz}
\usepackage{tcolorbox}
\usepackage[all]{xy}
\tcbuselibrary{skins, breakable, theorems}
\usepackage{enumerate}
%\usepackage{boondox-cal}
\usepackage{times}
\usepackage{bm}
\renewcommand{\baselinestretch}{1.4}

%\geometry{a4paper}%
%\geometry{left=3cm,right=2.5cm,top=2.5cm,bottom=2.5cm}  %设置 上、左、下、右 页边距
\pagestyle{plain}


%\newtcolorbox{mybox}{colback = red!25!white, colframe = red!75!black}
\definecolor{darkgreen}{rgb}{0,0.35,0}
\newtcolorbox{mybox}[2][]{width=10cm,colback = red!5!white, colframe = green!75!black, fonttitle = \bfseries,colbacktitle = red!55!yellow, enhanced,attach boxed title to top left={yshift=-2mm},	title=#2,#1}
%\newtcbtheorem{question}{}
%{enhanced, breakable,
%	colback = white, colframe = cyan, colbacktitle = cyan,
%	attach boxed title to top left = {yshift = -2mm, xshift = 5mm},
%	boxed title style = {sharp corners},
%	fonttitle = \sffamily\bfseries, separator sign = {).~}}{qst}



\theoremstyle{definition}

\newtheorem{thm}{{Theorem}\hspace{0.05pt}}[section]
\newtheorem{prop}[thm]{{Proposition}\hspace{0.05pt}}
\newtheorem{cor}[thm]{Corollary\hspace{0.05pt}}
\newtheorem{lem}[thm]{{Lemma}\hspace{0.05pt}}
\newtheorem{dfn}[thm]{{Definition}\hspace{0.05pt}}
\newtheorem{rem}[thm]{{Remark}\hspace{0.05pt}}
\newtheorem{exm}[thm]{{Example}\hspace{0.05pt}}



\newtheorem*{Q}{{Question}\hspace{0.05pt}}
\newtheorem*{A}{{Answer}\hspace{0.05pt}}
\newtheorem*{property}{{Property}\hspace{0.05pt}}



\newtheorem{conj}{Open Problem\hspace{0.05pt}}
\newtheorem*{prob}{Problem\hspace{0.05pt}}
\newtheorem*{fact}{{Fact}\hspace{0.05pt}}
\newtheorem{ex}{ \hspace{0.05pt}}[section]


\newtheorem*{cexm}{{Counterexample}\hspace{0.05pt}}

\newtheorem*{ntt}{{Notation}\hspace{0.05pt}}
%\newtheorem*{dfn*}{Definition\hspace{0.05pt}}
%\newtheorem{note}[thm]{Note}
\newtheorem*{pf}{Proof}
\newtheorem*{Solution}{Solution}
\usepackage[top=3.2cm,bottom=3.8cm,left=3.0cm,right=3.0cm]{geometry}
%\newcommand{\Spec}{\mathrm{Spec}\ }
%\newcommand{\Gal}{\mathrm{Gal}\ }
\newcommand{\q}{\mathfrak{q}}
\newcommand{\p}{\mathfrak{p}}
\newcommand{\m}{\mathfrak{m}}
%\newcommand{\a}{\mathfrak{a}}
\newcommand{\F}{\mathcal{F}}
\newcommand{\G}{\mathcal{G}}
%\newcommand{\H}{\mathcal{H}}
\newcommand{\Cond}{\text{Cond}}
\newcommand{\An}{\text{An}}
\newcommand{\Sp}{\text{Sp}}





\begin{document}
\large
\title{\vspace{-6em}\Large{An and Sp}}
\date{}
\author{何力(Li HE) 322101140}
\maketitle
\section{An and Sp}
\subsection{An and $\An_*$}
\dfn
Denote the category of all Kan complexes by $\text{Kan}$, and define an $\infty$-category
$\text{An}=N(\text{Kan})$, we call $\An$ the $\infty$-category of anima, and its objects are called anima.\\
Similarly, denote the category of all pointed Kan complexes by $\text{Kan}_*$, and define an $\infty$-category
$\text{An}_*=N(\text{Kan}_*)$, we call $\An_*$ the $\infty$-category of pointed anima, and its objects are called pointed anima.


\dfn 
For $X\in \An_*$, define $\Sigma X:=\text{cofib}(X\rightarrow 0)$ and 
$\Omega X:=\text{fib}(0\rightarrow X)$, i.e. we have a pushout square
\begin{equation*}
\xymatrix{
	X\ar[r] \ar[d]&0\ar[d]\\
	0^\prime \ar[r]& \Sigma X
}
\end{equation*}
and a pullback square
\begin{equation*}
\xymatrix{
	\Omega X\ar[r] \ar[d]&0\ar[d]\\
	0^\prime \ar[r]&  X
}
\end{equation*}
We call $\Sigma:\An_*\rightarrow \An_*$ the suspension functor and 
$\Omega: \An_*\rightarrow \An_*$ the loop space functor. 



\prop 
For $X, Y\in \An_*$, we have a natural equivalence:
$$
\text{Map}_{\An_*}(\Sigma X, Y)\simeq 
\text{Map}_{\An_*}(X,\Omega Y).
$$
\pf 
\begin{align*}
\text{Map}_{\An_*}(\Sigma X,Y)
&\simeq 
\text{Map}_{\text{Fun}(\Lambda^2_0,\An_*)}
(0\gets X\to 0, Y\gets Y\to Y)\\
&\simeq
\text{Map}_{\An_*}(0,Y)
\times_{\text{Map}_{\An_*}( X,Y)}
\text{Map}_{\An_*}( X,Y)
\times_{\text{Map}_{\An_*}( X,Y)}
\text{Map}_{\An_*}(0,Y)\\
&\simeq 
\text{Map}_{\text{Fun}(\Lambda^2_2,\An_*)}
(X\to X\gets X, 0\to Y\gets 0)\\
&\simeq
\text{Map}_{\An_*}( X,\Omega Y).
\end{align*}
\qed 



\rem
$\An_*$ is generated under colimits by $\{S^n: n\geq 0\}.$

\prop 
The $\infty$-category $\An$ admits all small limits and colimits.
\pf Ref Theorem[4.3.3.7].

\prop 
\begin{itemize}
	\item [(1)]The forgetful functor $\An_*\rightarrow \An $ commutes with filtered colimits.
	\item [(2)]In the $\infty$-category $\An$, filtered colimits commute with limits.
	\item [(3)]
	$\pi_0:\An\rightarrow \text{Set}$ commutes with all colimits, and therefore 
	$$\pi_n:\An_*\stackrel{\Omega^n}{\longrightarrow} \An_*\stackrel{\text{fgt}}{\longrightarrow} \An\stackrel{\pi_0}{\longrightarrow} \text{Set}$$
	commutes with filtered colimits, for any $n\geq 1$.
\end{itemize}









\newpage
\subsection{Sp}
\dfn 
A spectrum is 
$E=\{E_n,\delta_n: E_n\stackrel{\sim}{\rightarrow}\Omega E_{n+1} \}_{n\in\mathbb{Z}}$, where $E_n\in \An_*$, for all $n\in\mathbb{Z}$. We denote the $\infty$-category of all spectra by $\Sp$.\\

We have a pair of adjoint functors 
$(\Sigma^\infty,\Omega^\infty):\An_*\rightarrow \Sp$, here we mean 
$\Sigma^\infty\dashv \Omega^\infty.$ We can define them as follows:
\begin{align*}
\Omega^\infty:\Sp\rightarrow \An_*;\  
 \{E_n,\delta_n: E_n\stackrel{\sim}{\rightarrow}\Omega E_{n+1} \}_{n\in\mathbb{Z}}\mapsto E_0,
\end{align*}
and
$$
\Sigma^\infty:\An_*\rightarrow \Sp;\ 
X\mapsto\Sigma^\infty X=\{Q\Sigma^n X, Q\Sigma^n X\stackrel{\sim}{\rightarrow}\Omega Q\Sigma^{n+1} X\}_{n\in \mathbb{Z}}.
$$
Here, $Q:\An_*\rightarrow \An_*$ is defined by
$$
QX:=\text{colim}(X\rightarrow \Omega\Sigma X\rightarrow \Omega^2\Sigma^2 X\rightarrow\cdots).
$$
\rem
We have:
\begin{align*}
\Omega Q\Sigma X
&\simeq \Omega\ \text{colim}(\Sigma X \rightarrow \Omega\Sigma \Sigma X\rightarrow \Omega^2\Sigma^2 \Sigma X\rightarrow\cdots)\\
&\simeq \text{colim} ( \Omega\Sigma X\rightarrow \Omega^2\Sigma^2 X\rightarrow\cdots)\\
&\simeq QX.
\end{align*}

\exm 
\begin{itemize}
	\item [(1)]
	For $A\in\text{Ab}$, we define $HA\in \Sp$ to be
	$$
	\{K(A,n), \delta: K(A,n)\stackrel{\sim}{\rightarrow} K(A,n+1) \}_{n\geq 0}.
	$$
	In fact, $H:\text{Ab}\rightarrow \Sp$ is a fully faithful functor. And we have $\Omega^\infty HA=A.$
      \item [(ii)] The sphere spectrum is defined to be
$\mathbb{S}:=\Sigma^\infty S^0$, where $S^0\in \An_*.$ 
\end{itemize}






\dfn 
The suspension functor $\Sigma$ and loop space functor $\Omega$ on $\Sp$ are defined as follows:
$$
\Sigma:\Sp\rightarrow \Sp;\ 
E=\{E_n \}_{n\in\mathbb{Z}}
\mapsto 
\Sigma E=\{E_{n+1}\}_{n\in \mathbb{Z}},
$$
and
$$
\Omega:\Sp\rightarrow \Sp;\ 
E=\{E_n \}_{n\in\mathbb{Z}}
\mapsto 
\Omega E=\{E_{n-1}\}_{n\in \mathbb{Z}}.
$$
Hence from the definition we know that $\Sigma$ and $\Omega$ are inverse to each other, so we can denote by $\Omega=\Sigma^{-1}$ and $\Sigma=\Omega^{-1}.$

\dfn 
For $E=\{E_n,\delta_n: E_n\stackrel{\sim}{\rightarrow}\Omega E_{n+1} \}_{n\in\mathbb{Z}}\in \Sp$ and 
$F=\{F_n,\delta_n: F_n\stackrel{\sim}{\rightarrow}\Omega F_{n+1} \}_{n\in\mathbb{Z}}\in \Sp$, define
\begin{align*}
\text{Map}_\Sp(E,F)
:&=
\underset{n}{\text{lim}}
(\cdots \leftarrow \text{Map}_{\An_*}(E_0,F_0)\leftarrow
\text{Map}_{\An_*}(E_1,F_1)\leftarrow\cdots)\\
&=\underset{n}{\text{lim}}\ \text{Map}_{\An_*}(E_n,F_n).
\end{align*}


~\\
We next show that $\Sp$ has all limits and all colimits.



\prop
The $\infty$-category $\Sp$ admits all limits and  filtered colimits. More concretely, 
\begin{itemize}
	\item [(1)]
	Assume $E:I\rightarrow \Sp$ is a limit daigram, then $\underset{I}{\text{lim}}\ E$ exists, and is given by
		$$
		\{\underset{I}{\text{lim}}\ E(i)_n, \delta_n:
	\underset{I}{\text{lim}}\ E(i)_n\stackrel{\sim}{\rightarrow}
		\Omega\  \underset{I}{\text{lim}}\ E(i)_{n+1}\}.
		$$
		\item [(2)]
		Assume $E:I\rightarrow \Sp$ is a filtered colimit daigram, then $\underset{I}{\text{colim}}\ E$ exists, and is given by
		$$
		\{\underset{I}{\text{colim}}\ E(i)_n, \delta_n:
		\underset{I}{\text{colim}}\ E(i)_n\stackrel{\sim}{\rightarrow}
		\Omega\  \underset{I}{\text{colim}}\ E(i)_{n+1}\}.
		$$
\end{itemize}
\pf 
We have 
\begin{align*}
\text{Map}_\Sp(F, \{\underset{I}{\text{lim}}\ E(i)_n\})
&\simeq \underset{n}{\text{lim}}\ 
\text{Map}_{\An_*}(F_n, \underset{I}{\text{lim}}\ E(i)_n)\\
&\simeq \underset{n}{\text{lim}}\  \underset{I}{\text{lim}}\ 
\text{Map}_{\An_*}(F_n,E(i)_n)\\
&\simeq  \underset{I}{\text{lim}}\ 
\text{Map}_{\Sp}(F,E(i)),
\end{align*} 
 which means $\{\underset{I}{\text{lim}}\ E(i)_n\}$ is the limit of $E(i).$ \\
For the filtered colimit case, the proof is similar.
\qed
 




Since the functor $\Omega:\An_*\rightarrow \An_*$ commutes with all limits and filtered colimits, we can define the limits and filtered colimits in $\Sp$ pointwise. However, since $\Omega:\An_*\rightarrow \An_*$ is not commute with all colimits, we cannot define the colimits in $\Sp$ pointwise.

In order to show that $\Sp$ has all colimits, we introduce the concept of prespectrum.
\dfn 
\begin{itemize}
	\item [(1)]A prespectrum is 
	$E=\{E_n,\delta_n: \Sigma E_n \rightarrow  E_{n+1} \}_{n\in\mathbb{Z}}$, where $E_n\in \An_*$, for all $n\in\mathbb{Z}$. We denote the $\infty$-category of all prespectra by $\text{PSp}$.
	\item [(2)] For a prespectrum $\{E_n,\delta_n:\Sigma E_n \rightarrow  E_{n+1} \}_{n\in\mathbb{Z}}$, we define its associated spectrum to be
	$$
	\text{colim}
	(\Sigma^\infty E_0\rightarrow \Omega \Sigma^\infty E_1\rightarrow
	 \Omega^2 \Sigma^\infty E_2\rightarrow \cdots)
	 =\text{colim}(\Omega^n \Sigma^\infty E_n).
	$$
\end{itemize}

\exm 
For $X\in\An_*$, we have a prespectrum
$\{\Sigma^n X, \Sigma \Sigma^n X\rightarrow \Sigma^{n+1}X\}.$ Its associated spectrum is $\Sigma^\infty X$, since
$$
\text{colim}(\Omega^n \Sigma^\infty \Sigma^n X)
=\text{colim}(\Omega^n \Sigma^n\Sigma^\infty  X)
=\text{colim}(\Sigma^\infty  X)
=\Sigma^\infty  X.
$$

\lem 
For $E\in \Sp$, we have $E\simeq \text{colim}(\Omega^n \Sigma^\infty  E_n)$.
\pf 
For any $F\in \Sp$, 
\begin{align*}
\text{Map}_\Sp(\text{colim}\ \Omega^n \Sigma^\infty  E_n, F)
&\simeq \text{lim}\ \text{Map}_\Sp(\Omega^n \Sigma^\infty  E_n, F)\\
&\simeq \text{lim}\ \text{Map}_{\An_*}(  E_n,\Omega^\infty \Sigma^n F)\\
&\simeq \text{lim}\ \text{Map}_{\An_*}(  E_n,F_n)\\
&\simeq \text{Map}_\Sp(E, F),
\end{align*}
then by Yoneda, we have $E\simeq \text{colim}(\Omega^n \Sigma^\infty  E_n)$.
\qed

\prop 
$\Sp$ has all colimits.
\pf Assume $E:I\rightarrow \Sp$ is a colimit diagram, then we form a prespectrum
$\{\underset{I}{\text{colim}}\  E(i)_n, 
\delta_n: \Sigma \underset{I}{\text{colim}}\  E(i)_n\rightarrow 
\underset{I}{\text{colim}}\  E(i)_{n+1}\}.$ We claim that its associated spectrum is the colimit of the diagram $E:I\rightarrow \Sp:$
\begin{align*}
\text{Map}_\Sp(\underset{n}{\text{colim}}\ \Omega^n\Sigma^\infty \underset{I}{\text{colim}}\  E(i)_n, F )
&\simeq 
\underset{n}{\text{lim}}\ \text{Map}_\Sp(\Sigma^\infty \underset{I}{\text{colim}}\  E(i)_n,\Sigma^n F )\\
&\simeq 
\underset{n}{\text{lim}}\ \text{Map}_{\An_*}( \underset{I}{\text{colim}}\  E(i)_n,\Omega^\infty \Sigma^n F )\\
&\simeq 
\underset{n}{\text{lim}}\ \underset{I}{\text{lim}}\ \text{Map}_{\An_*}(  E(i)_n,F_n )\\
&\simeq 
\underset{I}{\text{lim}}\ \underset{n}{\text{lim}}\ \text{Map}_{\An_*}(  E(i)_n,F_n )\\
&\simeq 
\underset{I}{\text{lim}}\ \text{Map}_{\Sp}(  E(i),F).
\end{align*}
Hence $\underset{n}{\text{colim}}\ \Omega^n\Sigma^\infty \underset{I}{\text{colim}}\  E(i)_n\in \Sp $ is the colimit of the diagram $E:I\rightarrow \Sp$.
Therefore, $\Sp$ has all colimits.
\qed






~\\
\prop
For $X\in \An_*$, and $E\in \Sp$, we have a natural equivalence:
$$
\text{Map}_\Sp(\Sigma^\infty X, E)\simeq 
\text{Map}_{\An_*}(X,\Omega^\infty E). 
$$
\pf 
\begin{align*}
\text{Map}_\Sp(\Sigma^\infty X, E)
&\simeq  \underset{n}{\text{lim}}\ 
\text{Map}_{\An_*}((\Sigma^\infty X)_n,E_n)\\
&\simeq  \underset{n}{\text{lim}}\ 
\text{Map}_{\An_*}(Q\Sigma^n X,E_n)\\
&\simeq  \underset{n}{\text{lim}}\ 
\text{Map}_{\An_*}(\underset{m}{\text{colim}}\ \Omega^m\Sigma^{n+m} X,E_n)\\
&\simeq  \underset{n}{\text{lim}}\ \underset{m}{\text{lim}}\ 
\text{Map}_{\An_*}(\Omega^m\Sigma^{n+m} X,E_n)\\
&\simeq \text{Map}_{\An_*}(X, E_0)\\
&\simeq \text{Map}_{\An_*}(X,\Omega^\infty E). 
\end{align*}
\qed

\exm 
Assume $K\in\text{An}_*$ is finite, then we have
\begin{align*}
\text{Map}_\Sp(\Sigma^\infty K, \Sigma^\infty X)
&\simeq \text{Map}_{\An_*}( K, \Omega^\infty\Sigma^\infty X)\\
&\simeq \text{Map}_{\An_*}( K,\text{colim } \Omega^k\Sigma^k X)\\
&\simeq \text{colim } \text{Map}_{\An_*}( K,\Omega^k\Sigma^k X)\\
&\simeq \text{colim } \text{Map}_{\An_*}(\Sigma^k K,\Sigma^k X).
\end{align*}





\prop [Stability]
In the $\infty$-category $\Sp$, a square
\begin{equation*}
\xymatrix{
	X_0\ar[r] \ar[d]&X_1\ar[d]\\
	X_2 \ar[r]& X_{12}
}
\end{equation*}
 is a pullback iff it is a pushout.


\dfn
\begin{itemize}
	\item [(i)] For $E, F\in\Sp$, $[E,F]:=\pi_0\text{Map}_\Sp(E,F).$
	\item [(ii)] For $E\in \Sp$, define $\pi_n E:=[\Sigma^n\mathbb{S}, E]=\pi_n\text{Map}_\Sp(\mathbb{S},E).$
	\item [(iii)] $E\in \Sp$ is a connective spectrum, if $\pi_n E\simeq 0$, $\forall n<0.$ We denote the $\infty$-category of connective spectra by $\Sp_{\geq 0}.$
\end{itemize} 
~\\
For any $E\in\Sp$, we have
$$\text{Map}_\Sp(\mathbb{S},E)\simeq \text{Map}_{\An_*}(S^0,\Omega^\infty E)\simeq \Omega^\infty E.$$
So $\pi_n E\simeq \pi_n \text{Map}_\Sp(\mathbb{S},E)\simeq 
\pi_n \Omega^\infty E. $


\exm
For $X\in\An_*$, $\Sigma^\infty X\in \Sp$, then
\begin{equation*}
\pi_n(\Sigma^\infty X)=\left\{
\begin{aligned}
0 & &n<0\\
\pi_n^s(X)& &n\geq 0
\end{aligned}
\right.
\end{equation*}
\pf 
For $n\geq 0$, we have:
\begin{align*}
\pi_n(\Sigma^\infty X)
&=\pi_n(\Omega^\infty\Sigma^\infty X)
=\pi_n(\underset{k}{\text{colim}}\ \Omega^k\Sigma^k X)\\
&=\underset{k}{\text{colim}}\  \pi_n (\Omega^k\Sigma^k X)
=\underset{k}{\text{colim}}\  \pi_{n+k} (\Sigma^k X)
=\pi_n^s(X).
\end{align*}





\dfn 
\begin{itemize}
	\item [(i)]
	For $E, F\in\Sp$, define $\text{map}(E,F)\in \Sp$ as follows:
	$$
	\text{map}(E,F)
	:=\{\text{Map}_\Sp(E,\Sigma^n F), \delta_n:
	\text{Map}_\Sp(E,\Sigma^n F)\stackrel{\sim}{\rightarrow}
	\Omega \text{Map}_\Sp(E,\Sigma^{n+1} F)\}_{n\in\mathbb{Z}}.
	$$
	\item [(ii)] For $E, F\in\Sp$, define $E\otimes F\in \Sp$ as follows:
	$$
	E\otimes F:=\underset{n,m}{\text{colim}}\ \Omega^{n+m}\Sigma^\infty(E_n\wedge F_m).
	$$
\end{itemize}

\rem 
By definition, we know that for $E, F\in\Sp$, $E\otimes F\simeq F\otimes E$.



\prop 
For $E,F,K\in \Sp$, there is a natural equivalence:
$$
\text{Map}_\Sp(E\otimes F, K)\simeq \text{Map}_\Sp(E,\text{map}(F,K)).
$$
\pf 
\begin{align*}
\text{Map}_\Sp(E\otimes F, K)
&\simeq \text{Map}_\Sp(\underset{n,m}{\text{colim}}\ \Omega^{n+m}\Sigma^\infty(E_n\wedge F_m), K)\\
&\simeq \underset{n,m}{\text{lim}}\  \text{Map}_\Sp
(\Omega^{n+m}\Sigma^\infty(E_n\wedge F_m), K)\\
&\simeq \underset{n,m}{\text{lim}}\  \text{Map}_\Sp
(\Sigma^\infty(E_n\wedge F_m),\Sigma^{n+m} K)\\
&\simeq \underset{n,m}{\text{lim}}\  \text{Map}_{\An_*}
(E_n\wedge F_m,\Omega^\infty \Sigma^{n+m} K)\\
&\simeq \underset{n,m}{\text{lim}}\  \text{Map}_{\An_*}
(E_n,\text{Map}_{\An_*}(F_m,\Omega^\infty \Sigma^{n+m} K))\\
&\simeq \underset{n,m}{\text{lim}}\  \text{Map}_{\An_*}
(E_n,\text{Map}_{\An_*}(F_m,(\Sigma^n K)_m))\\
&\simeq \underset{n}{\text{lim}}\  \text{Map}_{\An_*}
(E_n,\underset{m}{\text{lim}}\ \text{Map}_{\An_*}(F_m,(\Sigma^n K)_m))\\
&\simeq \underset{n}{\text{lim}}\  \text{Map}_{\An_*}
(E_n,\text{Map}_{\Sp}(F,\Sigma^n K))\\
&\simeq \underset{n}{\text{lim}}\  \text{Map}_{\An_*}
(E_n,\text{map}(F,K)_n)\\
&\simeq  \text{Map}_{\Sp}
(E,\text{map}(F,K)).
\end{align*}
\qed


\prop 
For $X, Y\in \An_*$,
$$\Sigma^\infty X\otimes \Sigma^\infty Y\simeq
\Sigma^\infty(X\wedge Y)$$.
\pf For any $E\in\Sp$,
\begin{align*}
\text{Map}_{\Sp}(\Sigma^\infty X\otimes \Sigma^\infty Y,E)
&\simeq \text{Map}_{\Sp}(\Sigma^\infty X,\text{map}(\Sigma^\infty Y,E))\\
&\simeq \text{Map}_{\An_*}(X,\Omega^\infty \text{map}(\Sigma^\infty Y,E))\\
&\simeq \text{Map}_{\An_*}(X,\text{Map}_{\Sp}(\Sigma^\infty Y,E))\\
&\simeq \text{Map}_{\An_*}(X,\text{Map}_{\An_*}( Y,\Omega^\infty E))\\
&\simeq \text{Map}_{\An_*}(X\wedge Y,\Omega^\infty E)\\
&\simeq \text{Map}_{\An_*}(\Sigma^\infty (X\wedge Y),E),
\end{align*}
then by Yoneda's lemma, we have: 
$\Sigma^\infty X\otimes \Sigma^\infty Y\simeq
\Sigma^\infty(X\wedge Y).$
\qed 
























\end{document}